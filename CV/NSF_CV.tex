\documentstyle[11pt,aaspp4]{article}

%% latex JVcv;  dvips -o JVcv.ps JVcv; ps2pdf JVcv.ps    

\begin{document}

\def\heading{\noindent}
\def\ref{\vskip -0.05in \hangindent 2.5pc \hangafter 1}
\def\listing{\vskip -0.1in \hangindent 2.5pc \hangafter 1}
\def\Bline#1{\hskip 0.3in $\bullet$ #1 \\ \smallskip}
\def\Sline#1{\hskip 0.6in #1 \\}
\def\line#1{#1 \\}


\centerline{\bf Biographical Sketch for Jacob Vanderplas}

\heading {\bf  Professional Preparation}

\hskip -0.1in \begin{tabular}{llll}
  University of Washington &  Computer Science & Postdoc & 2013 - 2014\\
  University of Washington &  Astronomy        & Postdoc & 2012 - 2013 \\
  University of Washington &  Astronomy        & Ph.D.   & 2006 - 2012 \\
  University of Washington &  Astronomy        & M.Sc.   & 2006 - 2007 \\
  Calvin College           &  Physics          & B.Sc.   & 1999 - 2003 \\
\end{tabular}

\heading {\bf  Appointments and Teaching Experience}

\hskip -0.1in \begin{tabular}{lll}
  UW eScience Institute    &  Director of Open Software & 2017 - Present \\
  UW eScience Institute    &  Director of Research - Physical Sciences & 2014 - Present \\
  University of Washington &  NSF Postdoctoral Fellow & 2013 - 2014 \\
  University of Washington &  Post-doctoral Researcher & 2012 - 2013\\
  University of Washington &  Research Assistant & 2007 - 2012\\
  University of Washington &  Planetarium Coordinator & 2008 - 2010\\
  Mt. Hermon Outdoor Science School &  K-12 Environmental Educator & 2004 - 2006\\
\end{tabular}

\heading {\bf Selected Related Publications} 

\ref
 {\bf Jake Vanderplas}, Andrew Connolly, \v{Z}eljko Ivezi\'{c}, \& Alex Gray. {\it AstroML: Machine Learning for Astronomy in Astrophysics}. CIDU proceedings, 2012

\ref
 {\bf Jake Vanderplas}, Andrew Connolly, Bhuvnesh Jain, \& Mike Jarvis. {\it Interpolating Masked Weak Lensing Signals with Karhunen-Loeve Analysis}. ApJ 744:180, 2012.

\ref
 {\bf Jake Vanderplas}, Andrew Connolly, Bhuvnesh Jain, \& Mike Jarvis. {\it 3D Reconstruction of the Density Field: An SVD Approach to Weak Lensing Tomography}. ApJ 727:118, 2011.

\ref
 {\it LSST Science Collaboration LSST Science Book, Version 2.0}, 2009 arXiv:0912.0201

\ref
 {\bf Jake Vanderplas} \& Andrew Connolly. {\it Reducing the Dimensionality of Data: Locally Linear Embedding of Sloan Galaxy Spectra}. AJ 138:1365, 2009.

\heading {\bf Other Significant Publications}

\ref
V. Vikram, A. Cabre, B. Jain \& {\bf J. VanderPlas}. {\it Astrophysical Tests of Modified Gravity: the Morphology and Kinematics of Dwarf Galaxies}. JCAP 08:20, 2013

\ref
Bhuvnesh Jain \& {\bf Jake Vanderplas}. {\it Tests of Modified Gravity with Dwarf Galaxies}. JCAP 10:32, 2011.

\ref
Scott Daniel, Andrew Connolly, Jeff Schneider, {\bf Jake Vanderplas} \& Liang Xiong. Classification of Stellar Spectra with LLE. AJ 142:203, 2011.

\ref
Pedregosa, F.; Varoquaux, G.; Gramfort, A.; Michel, V.; Thirion, B.; Grisel, O.; Blondel, M.; Prettenhofer, P.; Weiss, R.; Dubourg, V.; {\bf Vanderplas, J.}; Passos, A.; Cournapeau, D.; Brucher, M.; Perrot, M.; Duchesnay, E. {\it Scikit-learn: Machine learning in Python}. Journal of Machine Learning Research, 12:2825, 2011

\ref
R. Kessler, A. Becker, D. Cinabro, {\bf J. Vanderplas}, \& 42 co-authors. {\it First-Year Sloan Digital Sky Survey-II Supernova Results: Hubble Diagram and Cosmological Parameters}. ApJ 703:1374, 2009.

\heading {\bf Synergistic Activities}

\listing
{\bf Open Source Contributions:} I have developed fast sparse matrix eigen-decomposition code and graphical analysis for the numerical package {\tt scipy}, and a number of optimized supervised and unsupervised learning and data visualization methods for the machine learning packages {\tt scikit-learn} and {\tt MDP-toolkit}. I created {\tt SciDB-py}, a Python interface to SciDB.  I have also written and contributed to astronomy-specific packages such as {\tt astroML} (Astronomy Machine Learning), and {\tt SNANA} (Fermilab's supernova analysis software).

\listing
{\bf Digital Planetarium:} From 2010-2011, I managed and coordinated the upgrade of the UW planetarium to a digital system based on the World Wide Telescope software. This was a joint project between the University of Washington and Microsoft Research. I have developed related educational tools for K-12 class visits as well as undergraduate astronomy courses. I have occasionally partnered with Microsoft as an astronomy expert at a variety of education and technology conferences around the country.

\listing
{\bf Science Communication Fellow:} I have participated in the Portal to the Public training program at the Pacific Science Center, and have volunteered regularly since 2009 as a Science Communication Fellow, exploring astronomical research with visitors to the museum.

\listing
{\bf Undergraduate Mentoring:} I have participated as a mentor for the U. Washington PreMajor in Astronomy Program (PreMAP), providing research experiences for under-graduates from demographics which are traditionally under-represented in the sciences.

\listing
{\bf K-12 Curriculum Development:} I taught for two years at the Mount Hermon Outdoor Science School, where among other activities I developed an outdoor astronomy curriculum for K-12 students, and conducted an astronomy training workshop for my peers at a regional conference for outdoor educators.

\end{document}
